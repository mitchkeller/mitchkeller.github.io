\documentclass{scrbook}
\KOMAoptions{twoside=false}
\usepackage{mathptmx,fullpage,amsmath,amssymb,amsthm}

%% Here are some custom commands relevant to MATH 315 and 416
\newcommand{\N}{\mathbb{N}} % \N gets you the "blackboard bold" N
                            % for natural numbers
\newcommand{\Z}{\mathbb{Z}} % \Z gets you the "blackboard bold" Z for
                            % integers 
\newcommand{\Q}{\mathbb{Q}} % \Q gets you the "blackboard bold" Q for
                            % rational numbers
\newcommand{\divides}{\mid} % If you want to write "a divides b"
% symbolically, you can use a\divides b.
\newcommand{\ideal}[1]{\left\langle\, #1 \,\right\rangle} % \ideal{x}
                                % gets you the notation for the
                                % ideal generated by x
\newcommand{\gengrp}[1]{\left\langle\, #1 \,\right\rangle} % \gengrp{x,y}
                                % gets you the notation for the
                                % group generated by x and y
\newcommand{\normal}{\trianglelefteq} % \normal gets you the normal
                                % subgroup symbol
\newcommand{\spin}[2]{\text{Spin}_{#1\times #2}} %\spin{3}{3} gets you
                                %the notation for Spin_{3\times 3}

\newenvironment{theorem}[2][Theorem]{\begin{trivlist}
\item[\hskip \labelsep {\bfseries #1}\hskip \labelsep {\bfseries #2.}]}{\end{trivlist}}
\newenvironment{lemma}[2][Lemma]{\begin{trivlist}
\item[\hskip \labelsep {\bfseries #1}\hskip \labelsep {\bfseries #2.}]}{\end{trivlist}}
\newenvironment{exercise}[2][Exercise]{\begin{trivlist}
\item[\hskip \labelsep {\bfseries #1}\hskip \labelsep {\bfseries #2.}]}{\end{trivlist}}
\newenvironment{problem}[2][Problem]{\begin{trivlist}
\item[\hskip \labelsep {\bfseries #1}\hskip \labelsep {\bfseries #2.}]}{\end{trivlist}}
\newenvironment{question}[2][Question]{\begin{trivlist}
\item[\hskip \labelsep {\bfseries #1}\hskip \labelsep {\bfseries #2.}]}{\end{trivlist}}
\newenvironment{corollary}[2][Corollary]{\begin{trivlist}
\item[\hskip \labelsep {\bfseries #1}\hskip \labelsep {\bfseries #2.}]}{\end{trivlist}}

\begin{document}
\title{MATH 416 Portfolio\\Spring 2022}
\author{John Q. Student}
\date{Spring 2022}
\maketitle
\tableofcontents
\chapter{Brief Essays}

\section{Growth as a mathematician}

% (2 or 3 paragraphs) Describe how you’ve improved as a mathematician
% and student of group theory this semester. Don’t just tell me what you
% learned---focus on one or two important and specific ways that you
% have changed or grown over the course of this semester. Identify at
% least one artifact in your portfolio that illustrates your growth
% area. Explain in the essay how this artifact shows growth in the way
% you’ve described.

Your response here.

\section{Definitions}

% (2 or 3 paragraphs): Give a clear, thorough, and general explanation
% of the role of definitions in mathematics. Include: What are
% definitions? Why are they necessary? Where do they come from? Are
% definitions ``true''? Identify an artifact that illustrates the
% importance and use of definition(s), and explain how they are used in
% it. Be specific.

Your response here.

  \section{Normal Subgroups and Quotient Groups}

  % (2 or 3 paragraphs): Clearly describe why the set of cosets of a
  % subgroup \(H\) of a group \(G\) can be viewed as a group when \(H\)
  % is a normal subgroup of \(G\). You can illustrate with an example,
  % perhaps from the group for your investigative project. Then give an
  % example of a subgroup \(H'\) that is not normal in a group \(G\) and
  % what goes wrong when trying to define a group operation on the
  % cosets. Multiplication tables might be helpful here, but you don't
  % need to include a full multiplication table. Be concrete and
  % specific!

  Your response here.

  \chapter{Final Grade Reflection}

  % This chapter is a self-evaluation of your performance in this
  % course. Your goal is to convincingly argue that you have met the
  % criteria for a certain grade. Be thoughtful, reflective, yet
  % concise: around 2 pages total.
  
  % Put a sentence or two here stating the grade you earned.

  \section{Requirements Met}

  % Find your grade’s description on the “How are course grades
  % determined?” section in the Course Policies and
  % Expectations. Explain how you have satisfied the appropriate grade
  % criteria. Be specific and
  % thorough. List each criterion you’ve met and how you know you’ve
  % met it. Refer to specific artifacts or examples in the portfolio
  % that support your case. Be brief and focused. I’m interested in
  % your reasoning.

  % Discussion of +/- attachments to your grade can also be included
  % here. Make your best case.

  \section{Unofficial Criteria}

% Are there important criteria that are missing? Explain why
%   they are useful criteria, and how well you met them. Again, be
%   specific, thorough, and refer to specific artifacts or examples
%   showing how you met them. 

  \section{Speed Round}
  % Turn each of the questions below into a one-sentence statement
  % that includes a brief explanation. Delete the question and replace
  % it with your response, but phrase so I can tell what the question
  % is. For example, "The most difficult part of the class for me was
  % ___ because ___."
  \begin{enumerate}
  \item What was the most difficult part of the class for you?
    You can list a general topic, a specific proof, a type of work we
    did, etc. Tell me why! 
  \item What was the easiest part of the class for
    you? Tell me why! 
  \item What part of class surprised or interested you the
    most? Why? One sentence!
  \end{enumerate}

  \chapter{Artifacts}

  % Carefully select items that show how you met specific grade
  % criteria. These will likely include many of your weekly homework
  % problems and contributions to the textbook, but can include other
  % items too. For example: Did you have great work on a daily prep,
  % but didn’t present it? Share your notes! If you gave a
  % presentation that helps show how you’ve met some criteria, include
  % a brief description of it (problem number, title, and when you
  % presented it). Many other things are possible. In the end,
  % anything that supports your argument for how you’ve earned a grade
  % is fair game! Please keep these focused and be selective: If
  % something (other than the expected components of your
  % investigative project report and slides) doesn’t directly address
  % one of the grade criteria, you don’t need to include it.

  % I encourage you to use \section{title} and \subsection{title} to
  % organize your artifacts! Put meaningful titles in that identify
  % what the artifact is, what criterion it addresses, or both!
  \section{A grouping of related artifacts}
  \subsection{The first artifact}
  \subsection{The second artifact}

  \section{A major artifact that is on its own}
  \chapter{Investigative Project}

  % The report for your investigative project goes here. Organize
  % using \section{title} and perhaps \subsection{title} as appropriate.


\end{document}
