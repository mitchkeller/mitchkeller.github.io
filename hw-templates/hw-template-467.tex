% --------------------------------------------------------------
% This is all preamble stuff that you don't have to worry about.
% Head down to where it says "Start here"
% --------------------------------------------------------------
% Lightly modified from Dana C. Ernst's homework template by Mitchel T. Keller

\documentclass[12pt]{amsart}
 
%\usepackage[margin=1in]{geometry} 
\usepackage{amsmath,amsthm,amssymb}
\usepackage{graphicx}
% If you have issues getting this document to compile BEFORE YOU EDIT
% ANYTHING, then I suggest commenting out the line below, which merely
% changes the document font.
\usepackage{mathptmx}


%% Here are some custom commands relevant to MATH 341
\newcommand{\N}{\mathbb{N}} % \N gets you the "blackboard bold" N
                            % for natural numbers
\newcommand{\Z}{\mathbb{Z}} % \Z gets you the "blackboard bold" Z for
                            % integers 
\newcommand{\R}{\mathbb{R}} % \R gets you the "blackboard bold" R for
                            % real numbers
\newcommand{\divides}{\mid}

\newenvironment{theorem}[2][Theorem]{\begin{trivlist}
\item[\hskip \labelsep {\bfseries #1}\hskip \labelsep {\bfseries #2.}]}{\end{trivlist}}
\newenvironment{lemma}[2][Lemma]{\begin{trivlist}
\item[\hskip \labelsep {\bfseries #1}\hskip \labelsep {\bfseries #2.}]}{\end{trivlist}}
\newenvironment{exercise}[2][Exercise]{\begin{trivlist}
\item[\hskip \labelsep {\bfseries #1}\hskip \labelsep {\bfseries #2.}]}{\end{trivlist}}
\newenvironment{problem}[2][Problem]{\begin{trivlist}
\item[\hskip \labelsep {\bfseries #1}\hskip \labelsep {\bfseries #2.}]}{\end{trivlist}}
\newenvironment{question}[2][Question]{\begin{trivlist}
\item[\hskip \labelsep {\bfseries #1}\hskip \labelsep {\bfseries #2.}]}{\end{trivlist}}
\newenvironment{corollary}[2][Corollary]{\begin{trivlist}
\item[\hskip \labelsep {\bfseries #1}\hskip \labelsep {\bfseries #2.}]}{\end{trivlist}}

\begin{document}
 
\title{Weekly Homework x\\MATH 467---Linear Algebra}
\author{Mitchel T.\ Keller}
\maketitle

\noindent\textbf{Collaboration statement}: This is where you put your
collaboration statement. Ensure that it makes specific reference to
this assignment's problems!

\begin{problem}{x} %You can use theorem, exercise, problem, or question here.  Modify x to be whatever number you are proving/solving
Delete this text and write the problem statement (or at least a
summary of it here. If the problem asks you to find a formula for
something and prove/explain it, restate the problem in a definitive
way that includes the formula/number you found, rather than as a
command of the form ``Find a formula\dots''. 
\end{problem}
 
\begin{proof}
Blah, blah, blah.  Here is an example of the \texttt{align} environment:
%Note 1: The * tells LaTeX not to number the lines.  If you remove the *, be sure to remove it below, too.
%Note 2: Inside the align environment, you do not want to use $-signs.  The reason for this is that this is already a math environment. This is why we have to include \text{} around any text inside the align environment.
\begin{align*}
\sum_{i=1}^{k+1}i & = \left(\sum_{i=1}^{k}i\right) +(k+1)\\ 
& = \frac{k(k+1)}{2}+k+1 & (\text{by inductive hypothesis})\\
& = \frac{k(k+1)+2(k+1)}{2}\\
& = \frac{(k+1)(k+2)}{2}\\
& = \frac{(k+1)((k+1)+1)}{2}.
\end{align*}
\end{proof}
 
\begin{theorem}{x.yz}
Let \(n\in \Z\).  Then yada yada.
\end{theorem}
 
\begin{proof}[Solution] %Notice that this prints "Solution" instead of "Proof"
  See the code for how I made this say ``Solution''

\end{proof}


\end{document}
