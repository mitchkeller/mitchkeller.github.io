\documentclass{amsart}

\usepackage{mathptmx}
\newcommand{\gengrp}[1]{\left\langle\, #1 \,\right\rangle} % \gengrp{x,y}
                                % gets you the notation for the
                                % group generated by x and y
\newcommand{\normal}{\trianglelefteq} % \normal gets you the normal
                                % subgroup symbol

%% Code block that needs to go into your preamble
%% (before \begin{document}

\usepackage{tikz}
\usetikzlibrary{arrows,automata,positioning}
\usepackage{color}
\definecolor{orange}{HTML}{E66101}
\definecolor{lightorange}{HTML}{FDB863}
\definecolor{purple}{HTML}{5E3C99}
\definecolor{lightpurple}{HTML}{B2ABD2}
\definecolor{green}{HTML}{4dAC26}
\definecolor{lightgreen}{HTML}{B8E186}
\definecolor{grey}{rgb}{.85, .85, .85}

%% End of code block for preamble
%%

\begin{document}
\title{MATH 416 Investigative Project\\Spring 2022}

\author{Mitchel T.\ Keller}
\maketitle

%% Code block that needs to be used to set up your arrows
%% For a double-headed arrow, use stealth-stealth, as with c-a and c-g
%% For a single-headed arrow, use -stealth as with c-m
%% Strongly recommend renaming things from c-a, c-m, and c-g to
%% match whatever your generators are, as it will make the coding easier.
\tikzstyle{vert} = [circle, draw, fill=grey,inner sep=0pt, minimum size=6.5mm]
\tikzstyle{c-a} = [draw,very thick,purple,stealth-stealth]
\tikzstyle{c-m} = [draw, very thick, orange,-stealth]
\tikzstyle{c-g} = [draw, very thick, green, stealth-stealth]

\begin{figure}[!ht]
\centering
\begin{tikzpicture}[scale=1.5,auto]
  %% \node draws vertices. In the first set of parentheses, put a
  %% meaningful name for the vertex. I used m, m2, and am3 type things
  %% since I am writing all my group elements as a^i m^j. In the
  %% second set of parentheses, you put the angle (0 is
  %% positive x-axis), a colon, and then the radius of the circle on
  %% which the vertex is placed.
\node (e) at (90:1.5) [vert] {{\scriptsize $e$}};
\node (m) at (135:1.5) [vert] {{\scriptsize $m$}};
\node (m2) at (180:1.5) [vert] {{\scriptsize $m^2$}};
\node (m3) at (225:1.5) [vert] {{\scriptsize $m^3$}}; 
\node (m4) at (270:1.5) [vert] {{\scriptsize $m^4$}}; 
\node (m5) at (315:1.5) [vert] {{\scriptsize $m^5$}}; 
\node (m6) at (0:1.5) [vert] {{\scriptsize $m^6$}}; 
\node (m7) at (45:1.5) [vert] {{\scriptsize $m^7$}};

\node (a) at (90:3) [vert] {{\scriptsize $a$}};
\node (am) at (135:3) [vert] {{\scriptsize $am$}};
\node (am2) at (180:3) [vert] {{\scriptsize $am^2$}};
\node (am3) at (225:3) [vert] {{\scriptsize $am^3$}}; 
\node (am4) at (270:3) [vert] {{\scriptsize $am^4$}}; 
\node (am5) at (315:3) [vert] {{\scriptsize $am^5$}}; 
\node (am6) at (0:3) [vert] {{\scriptsize $am^6$}}; 
\node (am7) at (45:3) [vert] {{\scriptsize $am^7$}};

%% Use \path to draw the arrows. Set up the arrow colors and types
%% earlier. If you renamed as suggested, then you'll need to change
%% the content of the square brackets here to be your arrow types. In
%% the parentheses, you are going to put the names of your
%% nodes/vertices. 
\path[c-m] (e) to (m);
\path[c-m] (m) to (m2);
\path[c-m] (m2) to (m3);
\path[c-m] (m3) to (m4);
\path[c-m] (m4) to (m5);
\path[c-m] (m5) to (m6);
\path[c-m] (m6) to (m7);
\path[c-m] (m7) to (e);

\path[c-m] (a) to (am7);
\path[c-m] (am7) to (am6);
\path[c-m] (am6) to (am5);
\path[c-m] (am5) to (am4);
\path[c-m] (am4) to (am3);
\path[c-m] (am3) to (am2);
\path[c-m] (am2) to (am);
\path[c-m] (am) to (a);

\path[c-a] (e) to (a); 
\path[c-a] (m) to (am);
\path[c-a] (m2) to (am2); 
\path[c-a] (m3) to (am3); 
\path[c-a] (m4) to (am4);
\path[c-a] (m5) to (am5); 
\path[c-a] (m6) to (am6);
\path[c-a] (m7) to (am7); 

\end{tikzpicture}
\caption{Cayley diagram for $K_M = \gengrp{a,m}$ .}
\label{fig:cayley-diagram}
\end{figure}


\begin{figure}[!ht]
  \centering
  %% This works just like for the earlier example, but I put things to
  %% have one big circle of all the vertices here rather than a
  %% smaller circle of 8 inside a larger circle. Other layouts are
  %% possible. Experiment!
\begin{tikzpicture}[scale=1.5,auto]
\node (e0) at (90:3) [vert] {{\scriptsize $0$}};
\node (e1) at (112.5:3) [vert] {{\scriptsize $1$}};
\node (e2) at (135:3) [vert] {{\scriptsize $2$}};
\node (e3) at (157.5:3) [vert] {{\scriptsize $3$}}; 
\node (34) at (180:3) [vert] {{\scriptsize $4$}}; 
\node (e5) at (202.5:3) [vert] {{\scriptsize $5$}}; 
\node (e6) at (225:3) [vert] {{\scriptsize $6$}}; 
\node (e7) at (247.5:3) [vert] {{\scriptsize $7$}};
\node (e8) at (270:3) [vert] {\scriptsize {$8$}}; 
\node (e9) at (292.5:3) [vert] {{\scriptsize $9$}}; 
\node (e10) at (315:3) [vert] {{\scriptsize $10$}}; 
\node (e11) at (337.5:3) [vert] {{\scriptsize $11$}};
\node (e12) at (0:3) [vert] {{\scriptsize $12$}};
\node (e13) at (22.5:3) [vert] {{\scriptsize $13$}};
\node (e14) at (45:3) [vert] {\scriptsize {$14$}};
\node (e15) at (67.5:3) [vert] {{\scriptsize $15$}};


\path[c-m] (e0) to (e1);

\path[c-a] (e0) to (e15);

\path[c-g] (e0) to (e8);

\end{tikzpicture}
\caption{Positioning 16 elements around a circle}
\label{fig:cayley-diagram}
\end{figure}


\end{document}
